\documentclass{article}

\usepackage{urdu-prose}


\begin{document}

\begin{center} \Large
	\LR{Shoaib Skit}
\end{center}

\urdufont		% Start rendering using urdufont

سَن ۱۹۸۸ کی ایک سنہری شام، شہر لاہور میں ایک گھرانے میں ایک لڑکا پیدا ہوا جس کے والدین نے اُس کا نام شعیب رکھا۔ اُن کی بے پناہ خوشی دیکھ کر قدرت نے اُن دونوں کو بچے کے حوالے سے ایک ایک خواہش دی۔ 

شعیب کی امی نے دعا کی کہ ''اللہ کرے میرے بیٹے کی شکل دلیپ کمار اور بال شاہ رُخ خان جیسے ہوں`` اِسی خواہش کی تکمیل کی خاطر وہ بچپن کی ہر گرمیوں میں شعیب کو گنجا کرا دیتی تھیں۔ \textit{\LR{Show photo of bald little Shoaib}}

شعیب کے ابو نے دعا کی کہ ''اللہ کرے میرے بیٹے کی balling وسیم اکرم کی طرح ہو۔``

اور کسی کے بارے میں یقین سے کچھ نہیں کہا جا سکتا لیکن شعیب یہ ضرور سمجھتا آیا ہے ہے کے یہ دعائیں پوری ہو گئیں۔

\rule{\textwidth}{1pt}

شییب کا بچپن ہنستے کھیلتے، پڑھتے لکھتے،  کھاتے پیتے اور \LR{VCR} پر اَن گِنت دفعہ \LR{Aladdin} دیکھتے ہوئے گزر رہا تھا کے اچانک ۱۹۹۶ کے آس پاس شعیب کو محبت ہو گئی۔ \LR{Play song: I am in love}. یہ یقیناً اُس کے ابو کی دعا کا نتیجہ تھا کیوں کے شعیب نے اپنے آپ کو وسیم اکرم کے عشق میں گرفتار پایا۔

باوجود اُس کے کہ شعیب اچھی خاصی \LR{Leg Break balling} کراتا تھا اب اُس کو یقین ہو گیا کہ وہ ایک \LR{right-handed} وسیم اکرم ہے۔ اِس بات میں ذرا برابر سچائی نہیں تھی۔

\LR{Re-enact a regular cricket session including the idotic eye-lash toss, the arguments, and Shoaib's desire to both ball and bat first.}

شعیب: \LR{Please} عابد بھائی مجھے پہلا \LR{over} کرانے دیں۔\\
عابد: لیکن تم نے پہلی \LR{batting} بھی تو کی تھی\\
شعیب: \LR{Pleeeaase}۔ مجھے پتا ہے میں اِنہیں \LR{out} کر دونگا\\
عابد: اچھا بھئی ٹھیک ہے۔ لیکن ایک شرط پر۔ تم \LR{fast balling} نہیں کرائو گے۔\\
شعیب: کیا؟ آپ دیکھیں میں انہیں کیسے \LR{bowled} کرتا ہوں اپنی \LR{fast balling} سے

اور پھر شعیب کو خوب مار پڑتی تھی جب تک کے وہ مجبوراً \LR{spin} پر تبدیل نہیں ہو جاتا تھا۔

\rule{\textwidth}{1pt}

جب شعیب ۱۶ سال کا ہوا تو اُس کا خاندان امریکہ منتقل ہو گیا۔ امریکہ جانے سے اُس کی زندگی پر سب سے بڑا اژر یہ پڑا کہ جہاں وہ گھر کے بنے ہوئے \LR{french fries}، پاپڑ اور آم کھاتا تھا وہ اب \LR{chips}، \LR{coke}، اور \LR{pizza} کھانے لگا جس کے نتیجے میں وہ انسان کم اور ایک مہوٹی بلی زیادہ لگنے لگا۔  

\LR{Dania, Aiman and Alina walk across the stage holding placards for the three foods, then flip the placards and walk back across showing the american variants. At the end someone \textbf{unveils} a poster of Shoaib's fat picture.}

\end{document}
